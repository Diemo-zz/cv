%%%%%%%%%%%%%%%%%%%%%%%%%%%%%%%%%%%%%%%%%
% Medium Length Graduate Curriculum Vitae
% LaTeX Template
% Version 1.1 (9/12/12)
%
% This template has been downloaded from:
% http://www.LaTeXTemplates.com
%
% Original author:
% Rensselaer Polytechnic Institute (http://www.rpi.edu/dept/arc/training/latex/resumes/)
%
% Important note:
% This template requires the res.cls file to be in the same directory as the
% .tex file. The res.cls file provides the resume style used for structuring the
% document.
%
%%%%%%%%%%%%%%%%%%%%%%%%%%%%%%%%%%%%%%%%%

%----------------------------------------------------------------------------------------
%	PACKAGES AND OTHER DOCUMENT CONFIGURATIONS
%----------------------------------------------------------------------------------------

\documentclass[margin, 10pt]{res} % Use the res.cls style, the font size can be changed to 11pt or 12pt here

\usepackage{helvet} % Default font is the helvetica postscript font
%\usepackage{newcent} % To change the default font to the new century schoolbook postscript font uncomment this line and comment the one above
\usepackage{hyperref}

\setlength{\textwidth}{5.4in} % Text width of the document

\begin{document}

%----------------------------------------------------------------------------------------
%	NAME AND ADDRESS SECTION
%----------------------------------------------------------------------------------------

\moveleft.5\hoffset\centerline{\large\bf Diarmaid de B\'urca} % Your name at the top
 
\moveleft\hoffset\vbox{\hrule width\resumewidth height 1pt}\smallskip % Horizontal line after name; adjust line thickness by changing the '1pt'

\moveleft.5\hoffset\centerline{Weserstra{\ss}e 13A, 10247, Berlin, Germany} % Your address
\moveleft.5\hoffset\centerline{\href{mailto:diarmaiddeburca@gmail.com}{diarmaiddeburca@gmail.com}}
\moveleft.5\hoffset\centerline{+49 15204133305}

%----------------------------------------------------------------------------------------

\begin{resume}

%----------------------------------------------------------------------------------------
%	OBJECTIVE SECTION
%----------------------------------------------------------------------------------------
 
% \section{OBJECTIVE}  
% 
% A position in the field of computers with special interests in business applications programming, information processing, and management systems. 


% Received a 1:1 degree from NUIG (78.38\% overall). \\
% {\sl Final year thesis}: {\sl Calculating the Schwarzschild Radius For A 
% Non-rotating Black Hole} Over the course of the thesis I learned tensorial mathematics, then applied this 
% to calculate the event horizon for a non-rotating black hole in flat space.\\
% {\sl Subjects covered}: Computational Physics, 
% Quantum Mechanics, Electromagnetism and Fluid Dynamics, Mathematical Modelling, Non-linear Systems, Applied Optics, 
% Solid State Physics and Partial Differential Equations}

%----------------------------------------------------------------------------------------
%	COMPUTER SKILLS SECTION
%----------------------------------------------------------------------------------------


 
 
%----------------------------------------------------------------------------------------
%	PROFESSIONAL EXPERIENCE SECTION
%----------------------------------------------------------------------------------------
 
\section{DEVELOPMENT EXPERIENCE}
{\bf Backend Engineer} \hfill Apr. 2018 - Present \\
\href{https://www.makersite.de/}{Makersite GmbH}, Berlin, Germany
\begin{itemize} \itemsep -2pt
 \item Developed RESTful APIs
 \item Maintained and developed Python codebase
 \item Worked as team lead for different projects
\end{itemize}


{\bf Data Scientist } \hfill Nov. 2017 - Mar. 2018 \\
\href{http://www.reyar.de}{Rey Analytical Research}, Cologne, Germany
\begin{itemize} \itemsep -2pt
\item Performed t-tests, z-tests, F-tests, ANOVA and linear modelling as required
\item Worked on a project to project basis
\item Developed models in SAS and R
\end{itemize}


{\bf Software Engineer} \hfill Feb. 2016 - Oct. 2017 \\
\href{http://www.alphagenes.roslin.ed.ac.uk/diarmaid-debruca/}{Roslin Institute}, University of Edinburgh
\begin{itemize} \itemsep -2pt
 \item Maintained and extended animal simulation codes
 \item Developed a graphical user interface for animal simulation codes
 \item Worked on parallisisation of different simulation codes
 \item Worked on developing a massively parallel mixed model solver
\end{itemize}

{\bf Intern }\hfill Sept. 2012\\
\href{http://www.realsim.ie/home}{RealSim}, Unit 110, Business Innovation Centre, Upper Newcastle, Galway, Ireland  
\begin{itemize} \itemsep -2pt
 \item \href{https://www.youtube.com/watch?v=MRwUUMN-JCU}{Developed a supernovae video using Unity3D} 
\end{itemize}

\section{OTHER EXPERIENCE}
{\bf Tutor} \hfill Sept. 2010 - Jan. 2016 \\
\href{http://www.nuigalway.ie/physics/}{Physics Department}, National University of Ireland, Galway (NUIG)

 
{\bf Outreach Member} \hfill Sept. 2010 - Jan. 2016 \\
\href{http://www.nuigalway.ie/science/}{Collage of Science}, NUIG


{\bf Tour Guide} \hfill Sept. 2012 - Nov. 2012 \\
CERN Exhibition, Leisureland, Galway
% \begin{itemize}
% \item Ran tours for the CERN exhibition while in Galway
% \end{itemize} 




%----------------------------------------------------------------------------------------
%	COMMUNITY SERVICE SECTION
%---------------------------------------------------------------------------------------- 

% \section{COMMUNITY \\ SERVICE}
% 
% Organized and directed the 1988 and 1989 Grand Marshall Week \\
% ``Basketball Marathon.'' A 24 hour charity event to benefit the Troy Boys Club. Over 250 people participated each year. 


%----------------------------------------------------------------------------------------
%	EDUCATION SECTION
%----------------------------------------------------------------------------------------

\section{EDUCATION}
{\bf Doctorate (Astrophysics),  \href{http://www.nuig.ie}{NUIG} }\hfill Sep. 2010 - Dec. 2015 \\
 Thesis: {\sl Synchrotron Emission from Isolated Neutron Stars}\\
Developed a new model of synchrotron emission from high ($>10^{5}$T) magnetic fields in the optical regime. Implemented 
this model in the Pulsar Reverse Engineering Code (POREC), a massively parallel code that simulated emission from the entire
open magnetosphere of pulsar.   Compared the results of the simulations to the observations of the Crab pulsar in order to 
restrict the volume of the magnetosphere from which emission could occur. Simulations showed support for the slot gap model
of pulsar emission over the outer gap or polar cap models.

 % \item Due to finish in Sept. 2014
% As part of the PhD a reverse Engineering model 
%  is used to calculate the emission from the region surrounding a pulsar for a range of different viewing angles 
%  and different rotation angles with respect to the magnetic field.   A best-fit to observations is then used in 
%  order to constrain pulsar theories.   During the PhD I have kept the code up to date, transferred to code from 
%  Fortran to CUDA, and developed a new model for Synchrotron emission from high Magnetic Fields \\}
{\bf Bachelor of Science(Physics and Applied Mathematics)} \hfill Sept. 2006 - May 2010 \\
Received a First Class Honours (78.38\%) degree from NUIG \\

%----------------------------------------------------------------------------------------
%	EXTRA-CURRICULAR ACTIVITIES SECTION
%----------------------------------------------------------------------------------------
%----------------------------------------------------------------------------------------

\section{EXTRA-CURRICULAR \\ ACTIVITIES} 
Organiser, March for Science Edinburgh \hfill Apr. 2017 \\
\href{http://www.socs.nuigalway.ie/society_profiles/view/447}{AstroSoc} Comittee Member \hfill Sept. 2012-Feb. 2016
\begin{itemize}\itemsep -2pt
 \item Won Best New Society at the 2014 Board of Irish College Societies Awards
 \item Won Best Departmental and Best New Society at NUIG Society Awards, 2014
\end{itemize}
\href{http://www.socs.nuigalway.ie/society_profiles/view/73}{Physics Soc}. Treasurer \hfill Sept. $2009$ - Sept. $2010$ \\
\href{http://www.socs.nuigalway.ie/society_profiles/view/25}{Chess Soc}. Treasurer \hfill Sept. 2008 - Sept. 2010\\
\href{http://www.clubs.nuigalway.ie/club/293/aikido}{Aikido Club} Secretary \hfill Sept. 2007 - Sept. 2009 \\
\href{http://www.socs.nuigalway.ie/society_profiles/view/42}{FanSci Soc}. Treasurer \hfill Sept. 2007 - Mar. 2008 \\
Committee Member for the Leitir M\'oir Youth Club \hfill Sept. 2006 - May. 2007

\section{SKILLS} 
{\bf Computational Languages:} 
\begin{itemize} \itemsep -2pt
 \item Expert in Fortran (2008)
 \item Fluent in Python
 \item Experience with SAS, Java, R and Maple
\end{itemize}

{\bf Personal Skills}
\begin{itemize} \itemsep -2pt
 \item Experience leading team development of projects
 \item Experience developing projects individually
 \item Capable of effective time management
\end{itemize}

{\bf Other skills}
\begin{itemize} \itemsep -2pt
\item Fluent in parallel programming (MPI, OMP)
\item Fluent in LaTeX
\item Experience with BLAS, LAPACK and SCALAPACK
\item Experienced in mathematical analyses
\item Experience with AWS
\end{itemize}

\section{AWARDS}
% \begin{itemize}\itemsep -1pt
%  \item Received PhD Scholarship from Science Dept. NUIG 2010 - 2014
%  \item Awarded title of University Scholar of NUIG 2006
%  \item Awarded title of Meritorious Winner for Mathematical Contest in Modelling (MCM) in 2010 and Successful Participant in 2009 
%  \item Awarded Irish Scholarship from NUIG in 2006
%  \item Completed courses on High-End Computing and Fortran from the Irish Centre for High End Computing (ICHEC) in 2011
%  \item Received ECDL in 2004 
% \end{itemize}
Awarded Meritorious Winner in Mathematical Contest in Modelling (MCM) \hfill $2010$ \\
Awarded Successful Participant in MCM \hfill  $2009$ \\
Awarded Irish Scholarship from NUIG \hfill $2006$ \\
Winner of Cumann Cois Fharrige \hfill $2006$ \\
Awarded University Scholar of NUIG \hfill $2006$

\section{COURSES}
Completed courses in Fortran and Parallel Computing \hfill $2012$ \\
{\it Basic Molecular Genetics For Bioinformaticians}   \hfill $19$-$20$ May $2016$\\
{\it Methods, Strategies and Tools to Generate, Analyse and}  \hfill  $3$-$7$ Jun. $2016$ \\
\hspace*{0.2in} {\it Incorporate Genomic Data into Livestock Breeding Programs} \\
{\it Advanced MPI} @EPCC \hfill $29$-$30$ Sept. $2016$ \\
{\it Evolutionary Quantitive Genetics} \hfill $31$ Oct. - $4$ Nov $2016$

\section{REFERENCES}
Prof. Andy Shearer \\
Physics Dept. NUIG, Ireland\\
Contact: +353 (0)91 493114 \hfill \href{mailto:andrew.shearer@nuigalway.ie}{andrew.shearer@nuigalway.ie}\\
\\
Gavin Duffy\\
RealSim Ltd, Business Innovation Centre, Upper Newcastle, Galway, Ireland \\
Contact: +353 (0) 91 493114 \hfill \href{mailto:gavin@realsim.ie}{gavin@realsim.ie}\\
\\
Dr. Matt Redman\\
Physics Dept. NUIG, Ireland\\
Contact: +353 (0) 91 493357 \hfill \href{mailto:matt.redman@nuigalway.ie}{matt.redman@nuigalway.ie}\\

\end{resume}
\end{document}